\documentclass[12pt]{article}
\usepackage{fullpage}

\begin{document}
\parskip=12pt
\parindent = 0pt

{\large \bf Algebra}
The matrix equation $A = L L^T$ can be expanded to a system of $N \times N$
block equations, one for each subblock $A_{IJ}$:
\[  A_{IJ} = \sum_{K=1}^{N} L_{IK} \cdot (L^T)_{KJ}
= \sum_{K=1}^{\min (I,J)} L_{IK} \cdot L_{JK}^T \]

We can compute the entries in $L$ by rearranging the equations corresponding to
the lower triangle as
\[ L_{IJ} \cdot L_{JJ}^T = A_{IJ} -
\sum_{K=1}^{J-1} L_{IK} \cdot L_{JK}^T \]

The diagonal blocks must satisfy
\[ L_{JJ} \cdot L_{JJ}^T = A_{JJ} -
\sum_{K=1}^{J-1} L_{JK} \cdot L_{JK}^T. \]

We solve this by computing $L_{JJ}$ as the Cholesky factor of $A_{JJ} -
\sum_{K=1}^{J-1} L_{JK} \cdot L_{JK}^T$.  This is possible when all of the
blocks in the $J$th block row of $L$ to the left of the diagonal are known.

The offdiagonal blocks ($I > J)$ in the  $J$th block column of $L$ satisfy
\[ L_{IJ} = ( A_{IJ} - \sum_{K=1}^{J-1} L_{IK} \cdot L_{JK}^T)
\cdot L_{JJ}^{-T}  \]
We ``apply the inverse'' by solving linear systems of equations; $L_{JJ}$ is
triangular, so this is straight-forward.

{\large \bf Dependencies}

\emph{The offdiagonal block} $L_{IJ}$ then depends on all blocks in
the $I$th block row of $L$ to the left of the diagonal being known, as
well as the diagonal block in the $J$th block row and column.

\emph{The diagonal block} $L_{JJ}$ depended on all the off-diagonal
blocks $L_{JK}$.  Combining these results, we saw that each of these
off-diagonal blocks $L_{JK}$ depends on the corresponding diagonal
block $L_{KK}$.  This shows that $L_{JJ}$ depends on all the preceding
diagonal blocks $L_{KK}, K = 1 \ldots J-1$.  Since each of these
blocks depends on the off-diagonal blocks to its left, $L_{JJ}$ cannot
be computed until all of the other blocks in the leading $J \times J$
triangle have been computed.

The offdiagonal block $L_{IJ}, I > J$ then depends on all of the blocks in the
leading $J \times J$ triangle and all of the blocks to its left in the
$I$th block row.  It is \emph{independent} of all of the other off-diagonal
blocks below the $J$th row.  This is the source of lots of parallelism.

{\large \bf Task Types}

The block factorization depends on three kinds of tasks:
\begin{itemize}
\item Factoring diagonal blocks (standard Cholesky)
  $L_{JJ} \cdot L_{JJ}^T = \widehat{A_{JJ}}$
\item Solving linear systems involving $L_{JJ}^{-T}$
\item computing the modification $A_{IJ} - L_{IK} \cdot L_{JK}^T$ for some
  particular triple of block indices $I, J, K$.
\end{itemize}
We can do these tasks in any order that respects the need for the arguments to
have been computed.  There are many such orders.  The process necessarily starts
by computing the Cholesky factor $L_{11}$ which depends on no other tasks.
Following this, any or all off-diagonal blocks in the first block column can be
computed.  These then enable all modifications of the form $A_{IJ} - L_{I1}
\cdot L_{J1}^T, J > 1, I \ge J$.

{\large \bf Scheduling Tasks}

A \emph{data-flow} implementation proceeds by performing these and all
subsequent operations whenever their input arguments are ready.  Block Cholesky
factorization can be implemented in a simple data-flow style in Chapel.  The
dataflow code uses \emph{sync} or \emph{single} variables to register the status
of each particular operation.  The variables used in the code are arrays of
variables, one entry for each $I,J$ block:
\begin{description}
  \item [all\_schur\_complement\_mods\_done ] a \emph{single} variable, initially
    empty, which becomes full only when all of the $J-1$ modifications $A_{IJ} -
    L_{IK} \cdot L_{JK}^T$ needed to form $\widehat{A_{IJ}}$ have been
    completed.

  \item [block\_computed] a second \emph{single} variable, initially empty, which
    becomes full when the factorization step or the block solve step for this
    block has finished, marking that the block is available for use in other
    operations.

    \item [mod\_lock] a \emph{sync} variable used to lock the block so that only one
      modification step on this block can be active at any time.

    \item [schur\_complement\_mods\_to\_be\_done] an ordinary variable, which
      holds the count of modifications yet to be done.  Each block $A_{I,J}$
      will be sujected to $J-1$ modifications in total.  The counter is
      decremented inside a critical section controlled by mod\_lock.
\end{description}

\end{document}
